\documentclass{article}
\usepackage[landscape]{geometry}
\usepackage{url}
\usepackage{multicol}
\usepackage{amsmath}
\usepackage{esint}
\usepackage{amsfonts}
\usepackage{tikz}
\usetikzlibrary{decorations.pathmorphing}
\usepackage{amsmath,amssymb}
\usepackage{pgfplots}
\pgfplotsset{compat=1.18}


\usepackage{colortbl}
\usepackage{xcolor}
\usepackage{mathtools}
\usepackage{amsmath,amssymb}
\usepackage{enumitem}
\usepackage{hyperref}
\makeatletter

\newcommand{\N}{\mathbb{N}}
\newcommand{\Z}{\mathbb{Z}}
\newcommand{\Q}{\mathbb{Q}}
\newcommand{\R}{\mathbb{R}}
\newcommand{\C}{\mathbb{C}}
\newcommand{\K}{\mathbb{K}}
\newcommand{\m}{\cdot}
\newcommand{\vect}[1]{\mathbf{#1}} 

\newcommand*\bigcdot{\mathpalette\bigcdot@{.5}}
\newcommand*\bigcdot@[2]{\mathbin{\vcenter{\hbox{\scalebox{#2}{$\m@th#1\bullet$}}}}}
\makeatother

\title{Analysis 2 Cheat Sheet}
\usepackage[brazilian]{babel}
\usepackage[utf8]{inputenc}

\advance\topmargin-.8in
\advance\textheight3in
\advance\textwidth3in
\advance\oddsidemargin-1.5in
\advance\evensidemargin-1.5in
\parindent0pt
\parskip2pt
\newcommand{\hr}{\centerline{\rule{3.5in}{1pt}}}
%\colorbox[HTML]{e4e4e4}{\makebox[\textwidth-2\fboxsep][l]{texto}
\begin{document}


\begin{multicols*}{3}

\tikzstyle{mybox} = [draw=black, fill=white, very thick,
    rectangle, rounded corners, inner sep=10pt, inner ysep=10pt]
\tikzstyle{fancytitle} =[fill=black, text=white, font=\bfseries]


Sei im folgenden $X$ ein normierter Vektorraum, $A \subseteq X$, $f:A\to \R$ eine Funktion und $\vect x^0 \in A$.\newline
%------------ Globales/Lokales Minimum/Maximum allgemein---------------
\begin{tikzpicture}
\node [mybox] (box){%
    \begin{minipage}{0.3\textwidth}
    $f$ hat in $\vect x^0$ ein \textcolor{red}{(strenges) globales Maximum}, wenn $\forall \vect x \in A\backslash \{\vect x^0\}: f(\vect x) \leq f(\vect x^0)$ \hspace{5pt} (bzw. \hspace{5pt} $f(\vect x) < f(\vect x^0)$)\\
    \newline
    $f$ hat in $\vect x^0$ ein \textcolor{red}{(strenges) globales Minimum}, wenn $\forall \vect x \in A\backslash \{\vect x^0\}: f(\vect x) \geq f(\vect x^0)$ \hspace{5pt} (bzw. \hspace{5pt} $f(\vect x) > f(\vect x^0)$)\\
    \newline
    $f$ hat in $\vect x^0$ ein \textcolor{red}{(strenges) lokales Maximum}, wenn $\exists \varepsilon > 0 \forall \vect x \in B_{\varepsilon}(\vect x^0) \cap A\backslash \{\vect x^0\}: f(\vect x) \leq f(\vect x^0)$ \hspace{5pt} (bzw. \hspace{5pt} $f(\vect x) < f(\vect x^0)$)\\
    \newline
    $f$ hat in $\vect x^0$ ein \textcolor{red}{(strenges) lokales Minimum}, wenn $\exists \varepsilon > 0 \forall \vect x \in B_{\varepsilon}(\vect x^0) \cap A\backslash \{\vect x^0\}: f(\vect x) \geq f(\vect x^0)$ \hspace{5pt} (bzw. \hspace{5pt} $f(\vect x) > f(\vect x^0)$)\\
    \newline
    \textbf{\textcolor{red}{!}} Jedes globale Minimum/Maximum ist auch ein lokales Minimum/Maximum.
    \end{minipage}
};

%------------ Globales/Lokales Minimum/Maximum Header ---------------------
\node[fancytitle, right=10pt] at (box.north west) {Globales/Lokales Minimum/Maximum};
\end{tikzpicture}

%------------ Min-Max Dualität allgemein---------------
\begin{tikzpicture}
\node [mybox] (box){%
    \begin{minipage}{0.3\textwidth}
    Wenn $f$ in $\vect x^0$ ein lokales Minimum (Maximum) hat, dann hat $-f$ in $\vect x^0$ ein lokales Maximum (Minimum).
    \end{minipage}
};

%------------ Min-Max Dualität Header ---------------------
\node[fancytitle, fill = blue, right=10pt] at (box.north west) {Min-Max Dualität};
\end{tikzpicture}

%------------ Kompakte Mengen allgemein---------------
\begin{tikzpicture}
\node [mybox] (box){%
    \begin{minipage}{0.3\textwidth}
    Wir nennen $A$ \textcolor{red}{kompakt}, wenn jede Folge in $A$ eine konvergente Teilfolge hat, deren Grenzwert in $A$ liegt.
    \end{minipage}
};

%------------ Kompakte Mengen Header ---------------------
\node[fancytitle, fill = black, right=10pt] at (box.north west) {Kompakte Mengen};
\end{tikzpicture}

%------------ Kompakte Mengen Satz allgemein---------------
\begin{tikzpicture}
\node [mybox] (box){%
    \begin{minipage}{0.3\textwidth}
    \begin{itemize}
        \item Ist $A$ kompakt, so ist $A$ abgeschlossen und beschränkt.
        \item Ist $A$ kompakt und $B \subseteq A$ abgeschlossen, so ist $B$ kompakt.
        \item (Satz von Heine-Borel) Für $A \subseteq \K^n$ gilt:\\
        $A$ ist kompakt $\iff A $ ist abgeschlossen und beschränkt
        \item Sei $Y$ ein normierter VR, $A$ kompakt und\\$f:A \to Y$ eine stetige Funktion, dann ist $f(A)$ kompakt in $Y$.
        \item Ist $A$ kompakt und $f$ stetig, so $\exists \vect x_m, \vect x_M \in A$, so dass $f$ in $\vect x_m$ ein globales Minimum und in $\vect x_M$ ein globales Maximum hat.
    \end{itemize}
    \end{minipage}
};

%------------ Kompakte Mengen Satz Header ---------------------
\node[fancytitle, fill = blue, right=10pt] at (box.north west) {Kompakte Mengen Sätze};
\end{tikzpicture}

%------------ Satz von Taylor allgemein---------------
\begin{tikzpicture}
\node [mybox] (box){%
    \begin{minipage}{0.3\textwidth}
    Ist $I \subseteq \R$ ein offenes Intervall und $x_0, x \in I$ und $x \neq a$. Ist $f \in C^{n+1}(I,\K)$ für $n \in \N_0$, so gilt:\\
    $f(x) = T_n(x,x_0) + R_n(x,x_0)$, wobei \\
    \[T_n(x,x_0) := \sum_{k=0}^n \frac{f^{(k)}(x_0)}{k!}(x-x_0)^k \] und
    \[R_n(x,x_0) := \int_{x_0}^x \frac{f^{(n+1)}(t)}{n!}(x-t)^n dt  \] Wir nennen $T_n(x,x_0)$ das \textcolor{red}{n-te Taylorpolynom an der Stelle $x_0$} und $R_n(x,x_0)$ das \textcolor{red}{n-te Restglied (in Integralform)}\\
    \newline
    Für $\K = \R$ gilt ausserdem: Es existiert ein $\theta$ im Intervall $(x,x_0)$ bzw. $(x_0,x)$, für das gilt:\\
    \[R_n(x,x_0) = \frac{f^{(n+1)}(\theta)}{(n+1)!}(x-x_0)^{(n+1)}\]\\
    Wir nennen das dann das \textcolor{red}{n-te Restglied (in Lagrangeform)}
    \end{minipage}
};

%------------ Satz von Taylor Header ---------------------
\node[fancytitle, fill = blue, right=10pt] at (box.north west) {Satz von Taylor in $\R$};
\end{tikzpicture}





\end{multicols*}

\end{document}